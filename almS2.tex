% two new environments for LaTeX
%
% Bayesian Statistics III/IV Michaelmas 2006, JE.
%
% aufgabe - for exercises
% loesung - for solutions
%
% usage:   put a \input{envi.tex} statement in the header of your
%          LaTeX-document, and then simply:
%
%          \begin{question} text... \end{question}
%    or    \begin{solution} text... \end{solution}
\usepackage{amsmath,amsthm}
\usepackage{amsfonts}
\usepackage{color}
\usepackage{colordvi}
%\usepackage{defs}
\fboxsep.3cm
\newlength{\breite}
\breite\textwidth
\addtolength{\breite}{-21.78842pt}

\newcommand{\reell}{{\rm I\hspace{-0.08cm} R}}
\renewcommand{\a}{\alpha}
\renewcommand{\b}{\beta}
\newcommand{\e}{\epsilon}
\newcommand{\var}{\mbox{Var}}
\newcommand{\hvb}{\hat{\v \beta}}
\newcommand{\tZ}{\tilde{\v Z}}
%\newcommand{\R}{\textsf{R} }

% from Allan
\renewcommand{\v}[1]{\boldsymbol{#1}}
\newcommand{\hvS}{\hat{\v \Sigma}}
\newcommand{\Hmwk}{[{\bf Hand in to be marked}]\;}


%\newcommand{\laspace}{\;}
%\newcommand{\mespace}{\:}
%\newcommand{\smspace}{\,}
%\newcommand{\intspace}{\mespace}
%\newcommand{\isp}{\intspace}

%\newcommand{\eg}{{\em e.g.\ }}
%\newcommand{\ie}{{\em i.e.\ }}
%\newcommand{\cf}{{\em cf.\ }}
%\newcommand{\etal}{{\em et al.\ }}
%\newcommand{\etalnodotnospace}{{\em et al}}
%\newcommand{\eqcomma}{\laspace,}
%\newcommand{\eqstop}{\laspace.}
%\newcommand{\eqsemi}{\laspace;}
%\newcommand{\eqquestion}{\laspace?}
%\newcommand{\noi}{\noindent}
%\newcommand{\real}{{\mathbb R}}


\newcounter{aufg}[section]
\newenvironment{question}%
  {\refstepcounter{aufg}\noindent\textbf{Question \arabic{section}.\arabic{aufg}:}
   \\*[1ex]\noindent}{\vspace{.5cm}}

\newenvironment{questandremark}[1]%
  {\refstepcounter{aufg}\noindent\textbf{Question \arabic{section}.\arabic{aufg}: #1}
   \\*[1ex]\noindent}{\vspace{.5cm}}

\newenvironment{questandremarkCC}[1]%
  {\refstepcounter{aufg}\noindent\textbf{Question \arabic{section}.\arabic{aufg}CC: #1}
   \\*[1ex]\noindent}{\vspace{.5cm}}



\newcommand{\mathead}[2]%
{\hrule
\vspace{.15cm}
\begin{minipage}{\textwidth}
\textsf{\textbf{Introduction To Statistics For Data Science \hfill
Handout  #1\\
Dr. Ric Crossman\hfill #2
}}%
\end{minipage}
\vspace{.05cm}
\hrule}

\newcommand{\head}[2]%
{\hrule \vspace{.15cm} \noindent\textsf{\textbf{Introduction To Statistics For Data Science}}\hfill
\textsf{\textbf{Workshop #1}}\\
\textsf{\textbf{Dr. Ric Crossman}}\hfill \textsf{\textbf{#2}}
%\textsf{\textbf{Dr. Ian Jermyn}}\hfill \textsf{\textbf{#2}}
\vspace{.2cm}
\hrule

\vspace{1cm}

}

\newcommand{\lshead}[2]%
{\hrule \vspace{.15cm} \noindent\textsf{\textbf{Introduction To Statistics For Data Science}}\hfill
\textsf{\textbf{Solutions to Sheet #1}}\\
%\textsf{\textbf{Dr. Ian Jermyn}}\hfill \textsf{\textbf{#2}}
\textsf{\textbf{Dr. Ric Crossman}}\hfill \textsf{\textbf{#2}}
\vspace{.2cm}
\hrule

\vspace{1cm}

}

%%%%%%%%%%%%%%%%%%%%%%%%
% Aufgabenumgebung NEU %
%%%%%%%%%%%%%%%%%%%%%%%%

\newcounter{auf}
\newenvironment{auf}%
{\refstepcounter{auf}
\begin{center}
\fcolorbox[gray]{0}{.95}{
\makebox[\breite]{
%\framebox[\textwidth]{
\textbf{Question \arabic{auf}}
%}\\*[1ex]\noindent
}}\\*[1ex]\noindent
\end{center}
}{\vspace{.5cm}}


%%%%%%%%%%%%%%%%%%%%%%%%%%%
% Nur ein kleiner Test... %
%%%%%%%%%%%%%%%%%%%%%%%%%%%


\newcounter{loes}[section]
\newenvironment{solution}%
{\stepcounter{loes}
\begin{center}
\fcolorbox[gray]{0}{.95}{
\makebox[\breite]{
\textbf{Solution \arabic{section}.\arabic{loes}}
}}\\*[1ex]\noindent
\end{center}
}{}


%%%%%%%%%%%%%%%%%%%%%%%%%%%%
% Noch ein kleiner Test... %
%%%%%%%%%%%%%%%%%%%%%%%%%%%%


\newenvironment{material}[1]%
{\begin{center}
\fcolorbox[gray]{0}{.95}{
\makebox[\breite]{
\textbf{To question #1}
}}\\*[1ex]\noindent
\end{center}\vspace{1cm}
}{\vspace{1cm}}

%%%%%%%%%%%%%%%%%%%%%%%%%%%%
% Noch ein kleiner Test... %
%%%%%%%%%%%%%%%%%%%%%%%%%%%%
%f�r ins Netz gestellte L�sungen

\newenvironment{ls}[1]%
{\begin{center}
\fcolorbox[gray]{0}{.95}{
\makebox[\breite]{
\textbf{Solution of Question #1}
}}\\*[1ex]\noindent
\end{center}\vspace{1cm}
}{\vspace{1cm}}

%%%%%%%%%%%%%%%%%%%%%%%%%%%%%%%%
% Und noch ein kleiner Test... %
%%%%%%%%%%%%%%%%%%%%%%%%%%%%%%%%

%Aufgabenumgebung f�r Klausuren, nummeriert die Aufgaben und �bernimmt Punktezahlen...

\newcounter{ka}
\newenvironment{ka}[1]% %Parameter = Punktezahl der Aufgabe
{\refstepcounter{ka}
{\it Name:..................................., Matrikelnummer:....................................}
\begin{center}
\framebox[\textwidth]{
\textbf{Question \arabic{ka}} \hfill #1 grades
}\\*[1ex]\noindent
\end{center}
%\vspace{\fboxsep}
}{\vspace{1cm}}


\newcounter{lka}
\newenvironment{lka}[1]% %Parameter = Punktezahl der Aufgabe
{\refstepcounter{lka}
\begin{center}
\framebox[\textwidth]{
\textbf{L\"osung \arabic{lka}} \hfill #1 grades
}\\*[1ex]\noindent
\end{center}
%\vspace{\fboxsep}
}{\vspace{1cm}}

% that's it.
